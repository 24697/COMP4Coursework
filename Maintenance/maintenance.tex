\chapter{System Maintenance}
\section{Environment}
\subsection{Software}
The software that I used to create my system can be found bellow:

\begin{itemize}
\item Python 3
\item IDLE
\item PyQt
\item SQLite 3
\end{itemize}
\subsection{Usage Explanation}
\subsubsection{Python 3}
I used Python 3 mainly because it's the only language that I know well enough to use for a project of this size.
\subsubsection{IDLE}
I used IDLE as it comes with python and it included syntax highlighting specificity for python. I am familiar with this highlighting so it is beneficial for me to use IDLE.
\subsubsection{PyQt}
I used PyQt as it works with python 3 and has integration with SQlite 3 which makes interacting the database much easer.
\subsubsection{SQLite 3}
I used SQLite 3 for my database as it was the only database language that I know, however it has good integration with python 3 and PyQt.
\subsubsection{SQL Inspector}
I used the SQL Inspector as it can help to quickly test and develop SQL code and view any database.

\subsection{Features Used}
\subsubsection{Python 3}
Some of the features that I used include the ability to import from other python files and packages e.g. PyQt.
\subsubsection{IDLE}
I chose to use IDLE as is has specific highlighting for python and I can run the code from IDLE to test it with out, complying it.
\subsubsection{PyQt}
PyQt was used as I found that because Python is an object oriented language it was easy to build the user interface in a way that was easy to understand while programming.
\subsubsection{SQLite 3}
I used SQLite 3 to create a relational database and to search across multiple tables.
\subsubsection{SQL Inspector}
I used the SQL Inspector "Browse Data" function during the development of the CLI interface to identify if my SQL code was working as expected. I also use the "Execute Query" function to help develop more SQL code.

\section{System Overview}

\subsection{System Component}

\section{Code Structure}
The general of the system is a model view controller structure with the MainWindow.py file acting as the controller (see page \pageref{fig:MW.py}), View.py file as the view (see page \pageref{fig:Con.py}) and the connection.py file as the model (see page \pageref{fig:View.py}). The view and model import many other files so that the individual parts could be devolved and tested before they were implemented in to the main system.

\subsection{Particular Code Section}

\section{Variable Listing}


\begin{longtable}{|l|l|l|p{4cm}|}
	\hline
	NAME           & DATA TYPE & EXAMPLE DATA          & DESCRIPTION                                                                                                \\ \hline
	current\_table & integer   & 5                     & hold the value of the current table as an integer indexing from 0                                          \\ \hline
	forename       & string    & Peter                 & used to hold the forename variable before it is added to the database                                      \\ \hline
	surname        & string    & Millard               & used to hold the surname variable before it is added to the database                                       \\ \hline
	club           & string    & Team Cambridge        & used to hold the club name variable before it is added to the database                                     \\ \hline
	hold           & any       & \%bridge\%            & used to hold any data temporally be for it is used further in the program                                  \\ \hline
	ok             & boolean   & True                  & hold the outcome of .prepare function of a SQL query shows if a query was prepared correctly for debugging \\ \hline
	data           & list      & ["peter","Millard",1] & used to hold the SQL query data before it is inserted in to the query                                      \\ \hline
	fast\_time     & sting     & 00:25:24              & holds the fastes time taken from the database when calculating the handicap(in the format of HH:MM:SS)     \\ \hline
	fast\_hour     & integer   & 00                    & holds the hour component of the fastes time                                                                \\ \hline
	fast\_min      & integer   & 25                    & holds the minuet component of the fastes time                                                              \\ \hline
	fast\_sec      & integer   & 24                    & holds the second component of the fastes time                                                              \\ \hline
	dif\_min       & integer   & 08                    & holds the minuet value after the 17 minuets have been take from the fast time                              \\ \hline
	hour\_min      & integer   & 00                    & holds the hour value converted to minuets                                                                  \\ \hline
	full\_min      & integer   & 08                    & hold the sum of the dif\_min and hour\_min values                                                          \\ \hline
	min\_sec       & integer   & 48                    & hold the value of full\_min converted in to seconds                                                        \\ \hline
	full\_sec      & integer   & 72                    & holds the sum of min\_sec and fast\_sec                                                                    \\ \hline
\end{longtable}

\section{System Evidence}

\clearpage
\subsection{User Interface}


\subsubsection{Club UI screens}

\begin{figure}[H]
\includegraphics{./Maintenance/UI/Club.png}
\caption{The Club UI} \label{fig:club_UI}
\end{figure}

\begin{figure}[H]
\includegraphics{./Maintenance/UI/ClubAD.png}
\caption{The Club Add Data UI} \label{fig:ClubAD_UI}
\end{figure}

\begin{figure}[H]
\includegraphics{./Maintenance/UI/ClubSearch.png}
\caption{The Club Search UI} \label{fig:ClubSearch_UI}
\end{figure}


\subsubsection{Club Reference UI Screens}

\begin{figure}[H]
\includegraphics{./Maintenance/UI/ClubRef.png}
\caption{The Club Reference UI} \label{fig:ClubRef_UI}
\end{figure}

\begin{figure}[H]
\includegraphics{./Maintenance/UI/ClubRefAD.png}
\caption{The Club Reference Add Data UI} \label{fig:ClubRefAD_UI}
\end{figure}

\begin{figure}[H]
\includegraphics{./Maintenance/UI/ClubRefSearch.png}
\caption{The Club Reference Search UI} \label{fig:ClubRefSearch_UI}
\end{figure}


\subsubsection{Course UI Screens}

\begin{figure}[H]
\includegraphics{./Maintenance/UI/Course.png}
\caption{The Course UI} \label{fig:Course_UI}
\end{figure}

\begin{figure}[H]
\includegraphics{./Maintenance/UI/CourseAD.png}
\caption{The Course Add Data UI} \label{fig:CourseAD_UI}
\end{figure}

\begin{figure}[H]
\includegraphics{./Maintenance/UI/CourseSearch.png}
\caption{The Course Search UI} \label{fig:CourseSearch_UI}
\end{figure}


\subsubsection{Event UI Screens}

\begin{figure}[H]
\includegraphics{./Maintenance/UI/Event.png}
\caption{The Event UI} \label{fig:Event_UI}
\end{figure}

\begin{figure}[H]
\includegraphics{./Maintenance/UI/EventAD.png}
\caption{The Event Add Data UI} \label{fig:EventAD_UI}
\end{figure}

\begin{figure}[H]
\includegraphics{./Maintenance/UI/EventSearch.png}
\caption{The Event Search UI} \label{fig:EventSearch_UI}
\end{figure}


\subsubsection{Event Points UI Screens}

\begin{figure}[H]
\includegraphics{./Maintenance/UI/EventPoints.png}
\caption{The Event Points UI} \label{fig:EventPoints_UI}
\end{figure}

\begin{figure}[H]
\includegraphics{./Maintenance/UI/EventPointsAD.png}
\caption{The Event Points Add Data UI} \label{fig:EventPointsAD_UI}
\end{figure}

\begin{figure}[H]
\includegraphics{./Maintenance/UI/EventPointsSearch.png}
\caption{The Event Point Search UI} \label{fig:EventPointsSearch_UI}
\end{figure}



\subsubsection{Event Reference UI Screens}

\begin{figure}[H]
\includegraphics{./Maintenance/UI/EventRef.png}
\caption{The Event Reference UI} \label{fig:EventRef_UI}
\end{figure}

\begin{figure}[H]
\includegraphics{./Maintenance/UI/EventRefAD.png}
\caption{The Event Reference Add Data UI} \label{fig:EventRefAD_UI}
\end{figure}

\begin{figure}[H]
\includegraphics{./Maintenance/UI/EventRefSearch.png}
\caption{The Event Reference Search UI} \label{fig:EventRefSearch_UI}
\end{figure}

\clearpage
\subsubsection{Event Type UI Screens}

\begin{figure}[H]
\includegraphics{./Maintenance/UI/EventType.png}
\caption{The Event Type UI} \label{fig:EventType_UI}
\end{figure}

\begin{figure}[H]
\includegraphics{./Maintenance/UI/EventTypeAD.png}
\caption{The Event Type Add Data UI} \label{fig:EventTypeAD_UI}
\end{figure}

\begin{figure}[H]
\includegraphics{./Maintenance/UI/EventTypeSearch.png}
\caption{The Event Type Search UI} \label{fig:EventTypeSearch_UI}
\end{figure}

\begin{landscape}
\subsubsection{Record UI Screens}

\begin{figure}[H]
\includegraphics{./Maintenance/UI/Record.png}
\caption{The Record UI} \label{fig:Record_UI}
\end{figure}
\end{landscape}
\begin{figure}[H]
\includegraphics{./Maintenance/UI/RecordAD.png}
\caption{The Record Add Data UI} \label{fig:RecordAD_UI}
\end{figure}

\begin{figure}[H]
\includegraphics{./Maintenance/UI/RecordSearch.png}
\caption{The Record Search UI} \label{fig:RecordSearch_UI}
\end{figure}


\subsubsection{Rider UI Screens}

\begin{figure}[H]
\includegraphics{./Maintenance/UI/Rider.png}
\caption{The Rider UI} \label{fig:Rider_UI}
\end{figure}

\begin{figure}[H]
\includegraphics{./Maintenance/UI/RiderAD.png}
\caption{The Rider Add Data UI} \label{fig:RiderAD_UI}
\end{figure}

\begin{figure}[H]
\includegraphics{./Maintenance/UI/RiderSearch.png}
\caption{The Rider Search UI} \label{fig:RdierSearch_UI}
\end{figure}


\subsubsection{Other UI Screens}

\begin{figure}[H]
\includegraphics{./Maintenance/UI/DataDrop.png}
\caption{The Data Drop Down Menu UI} \label{fig:DataDrop_UI}
\end{figure}

\begin{figure}[H]
\includegraphics{./Maintenance/UI/Delete.png}
\caption{The Delete UI} \label{fig:Delete_UI}
\end{figure}

\begin{figure}[H]
\includegraphics{./Maintenance/UI/Error.png}
\caption{The Error UI} \label{fig:Error_UI}
\end{figure}

\begin{figure}[H]
\includegraphics[width=\textwidth]{./Maintenance/UI/FileDia.png}
\caption{The File Dialogue UI} \label{fig:FileDia_UI}
\end{figure}

\begin{figure}[H]
\includegraphics{./Maintenance/UI/TableDrop.png}
\caption{The Table Drop Down UI} \label{fig:TableDrop_UI}
\end{figure}


\clearpage
\subsection{ER Diagram}
\begin{figure}[H]
\includegraphics[width=\textwidth]{./Maintenance/ER.pdf}
\caption{The ER Diagram of the final database} \label{fig:ER_Fin}
\end{figure}

\subsection{Database Table Views}
\clearpage

\begin{landscape}
\subsection{Database SQL}
\pythonfile{./Implementation/DatabaseConstructor.py}
\end{landscape}
\subsection{SQL Queries}

SQL code to find the fastest time of a given rider for the handicap calculation of a circuit event:

\pythonfile[firstline=33,lastline=43]{./Implementation/HandicapCal.py}

This queries uses nested queries to find the fasted time of a rider on the E33/10 course, as when calculating the handicap for a circuit event instead for using the fasted time for the same course the fasted time on a E33/10 event is used instead. The primary query is to find records ordered by time, I then used a nested queries to filter the records by EventID and then the EventID is ferther filtered by a second nested queries filtering them by CourseCode. Finally the results are ordered by ride time and the fist in the list should be the fastest time.

SQL code to find the fastest time of a given rider for the handicap calculation of a 10 mile event:

\pythonfile[firstline=52,lastline=62]{./Implementation/HandicapCal.py}

As before I have used nested queries for filtering across multiple tables. This query is used to find the fasted time for a rider on a 10 mile event. 

SQL code to find the fastest time of a given rider for the handicap calculation of a 25 mile event:

\pythonfile[firstline=71,lastline=81]{./Implementation/HandicapCal.py}
As before I have used nested queries for filtering across multiple tables. This query is used to find the fasted time for a rider on a 25 mile event. 
\section{Testing}

\subsection{Summary of Results}

\subsection{Known Issues}
\begin{itemize}
\item When adding data to the Event table the first time adding data will fail, you will need to resubmit the data and it shall be added as expected that time and all subsequent times.
\item When adding data to the Event table the Laps and CourseID get swapped, this can cause a compound error when trying to calculate a handicap modifier.
\end{itemize}
\section{Code Explanations}

\subsection{Difficult Sections}

\subsection{Self-created Algorithms}
The main element of the program is the handicap calculation, this is based of a modified version of the Cycling Time Trials (CTT) handicap method, the current Team Cambridge version of this uses a table to fin the handicap modifier. The general method for this is t find the riders fastest time for the same distance of the event of which the handicap is being calculated for and use a look up table (see figure \ref{fig:Specification} on page \pageref{fig:Specification}) to find the modifier, this modifier is then subtracted from the event ride time.

I had to create 3 different search SQL for each type of handicap statements that I have disused in section 4.5.5, however I also need to create separate calculations for each type of handicap.

\pythonfile[firstline=106,lastline=154]{./Implementation/HandicapCal.py}

For the 10 mile version of the calculation I first split the sting taken from the database in to the composite parts of hours, minuets and seconds, then the minuet modifier is taken from the original minuets value(the minuets modifier if the only part of the calculation that is altered between each distance, for a 10 mile event the value is 17, for a 25 event the value is 45 and for a circuit event the value is 11 minuets). Next the hour, minuet and seconds components are converted down to seconds, this value is then floor divided by 15 to find the "change" value which is taken from the "full\_sec" value to find the handicapped time. Finally the handicapped time is converted back to hours, minuets and seconds then reformatted to HH:MM:SS.
\section{Settings}
There are no specific setting for my system to run.
\section{Acknowledgements}

\section{Code Listing}

\begin{landscape}
\label{fig:MW.py}
\pythonfile{./Implementation/MainWindow.py} 

\subsection{Connection.py}
\label{fig:Con.py}
\pythonfile{./Implementation/Connection.py} 

\subsection{View.py}
\label{fig:View.py}
\pythonfile{./Implementation/View.py} 

\subsection{ErrorDialog.py}
\pythonfile{./Implementation/ErrorDialog.py}

\subsection{CreatingNewRecords.py}
\pythonfile{./Implementation/CreatingNewRecords.py}

\subsection{Search.py}
\pythonfile{./Implementation/Search.py}

\subsection{DialogBox.py}
\pythonfile{./Implementation/DialogBox.py}

\subsection{RecordDialogBox.py}
\pythonfile{./Implementation/RecordDialogBox.py}

\subsection{HandicapCal.py}
\pythonfile{./Implementation/HandicapCal.py}

\subsection{CLI.py}
\pythonfile{./Implementation/CLI.py}

\subsection{Model.py}
\pythonfile{./Implementation/Model.py}

\subsection{DleateData.py}
\pythonfile{./Implementation/DleateData.py}

\subsection{EditClub.py}
\pythonfile{./Implementation/EditClub.py}

\subsection{EditRider.py}
\pythonfile{./Implementation/EditRider.py}

\subsection{EditEventType.py}
\pythonfile{./Implementation/EditEventType.py}

\subsection{EditCourse.py}
\pythonfile{./Implementation/EditCourse.py}

\subsection{EditEventReference.py}
\pythonfile{./Implementation/EditEventReference.py}

\subsection{EditEvent.py}
\pythonfile{./Implementation/EditEvent.py}

\subsection{EditRecord.py}
\pythonfile{./Implementation/EditRecord.py}

\subsection{EditClubReference.py}
\pythonfile{./Implementation/EditClubReference.py}

\subsection{EditEventPoints.py}
\pythonfile{./Implementation/EditEventPoints.py}

\subsection{DatabaseConstructor.py}
\pythonfile{./Implementation/DatabaseConstructor.py}
\end{landscape}
