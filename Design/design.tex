\chapter{Design}

\section{Overall System Design}

\subsection{Short description of the main parts of the system}

Cycling database management system
\begin{itemize}
	\item Data management and general interface
	\item Creating new records in the database
	\item Exporting data on events
\end{itemize}

Data management and general interface
\begin{itemize}
	\item In the program there will need to be functionality that allows the user to search and sort the records.
	\item The program will need to be able to edit and delete existing records
	\item The program will need to be able to create new records
	\item The program will need to be able to view related information of the currently view record
	\item There will need to be keyboard short cuts as the program need to speed up the current system
\end{itemize}

Creating new records in the database
\begin{itemize}
	\item Opens a dialogue box forcing the user to enter key data to create the event
	\item The dialogue box will have a "Save" and "Save and create record" buttons as when the user wants to add an event they will almost always want to also add rider records
	\item when a new record is made the main layout will change with spaces for the user to enter the required data. There will also be buttons for "Saving" the record and "Saving and creating a new record" as these will be the two options that the user will need when creating 
\end{itemize}

Exporting data on events
\begin{itemize}
	\item The program will need to be able to export the data so that it can be uploaded to the club website and viewed by the public
	\item The data will need to be exported in csv format so that the website can properly format the data in a way that is easy to understand.
\end{itemize}
\subsection{System flowcharts showing an overview of the complete system}



\section{User Interface Designs}
\begin{figure}[H]
	\includegraphics[width=\textwidth]{./UIDesign/BaseMainDrop.pdf}
	\caption{The main layout of the program} \label{fig:The main layout of the program} 	
\end{figure}

\begin{figure}[H]
	\includegraphics[width=\textwidth]{./UIDesign/NewEventDialog.pdf}
	\caption{The new event dialogue box} \label{fig:The new event dialogue box}
\end{figure}
\section{Hardware Spesifiacation}

CPU:Pentium
RAM:4GB
HDD: 250Gb

\section{Program Structure}

\subsection{Top-down design structure charts}

\subsection{Algorithms in pseudo-code for each data transformation process}

\subsection{Object Diagrams}

\subsection{Class Definitions}

\section{Prototyping}

\section{Definition of Data Requirements}

\subsection{Identification of all data input items}

\subsection{Identification of all data output items}

\subsection{Explanation of how data output items are generated}

\subsection{Data Dictionary}

\begin{tabular}{|p{1.5cm}|p{1.5cm}|l|l|l|p{2.5cm}|}
	\hline
	NAME & DATA TYPE & LENGTH & VALIDATION & EXAMPLE DATA & APPROXIMATE SIZE \\ \hline
	EventID & & & & & \\ \hline
	Age & & & & & \\ \hline
	Date & & & & & \\ \hline
	EventPoints & & & & & \\ \hline
	EventType & & & & & \\ \hline
	EventPointType & & & & & \\ \hline
	ClubReferance & & & & & \\ \hline
	RiderID & & & & & \\ \hline
	Forename & & & & & \\ \hline
	Surname & & & & & \\ \hline
	HandicapMod & & & & & \\ \hline
	EventPointsID & & & & & \\ \hline
	Club & & & & & \\ \hline
	RecordID & & & & & \\ \hline
	RideTime & & & & & \\ \hline
	CourceID & & & & & \\ \hline
	CourceCode & & & & & \\ \hline
	ourceDistance & & & & & \\ \hline
	DateJoined & & & & & \\ \hline
	DateLeft & & & & & \\ \hline
	EventReferanceID & & & & & \\ \hline
	EventTypeID & & & & & \\ \hline
	ClubID & & & & & \\ \hline
	Laps & & & & & \\ \hline
\subsection{Identification of appropriate storage media}

\section{Database Design}

\subsection{Normalisation}

\subsubsection{ER Diagrams}
\begin{figure}[H]
    \includegraphics[width=\textwidth]{./ER/ERDesing.pdf}
\end{figure}

\subsubsection{Entity Descriptions}

Event(\underline{EventID}, \emph{CourseID} , \emph{EventReferanceID}, Date, Laps )

Course(\underline{CourseID}, CourseCode, CourseDistance)

Event Referance(\underline{EventReferanceID}, \emph{EventTypeID})

Event Type(\underline{EventTypeID}, EventType)

Rider(\underline{RiderID}, Forename, Surname)

Club Reference(\underline{ClubReferance}, \emph{RiderID}, \emph{ClubID}, DateJoined, DateLeft)

Club(\underline{ClubID}, Club)

Record(\underline{RecordID}, \emph{EventID}, RdieTime, Age, HandicapMod)

Event Points(\underline{EventPointsID}, \emph{RecordID}, EventPointsType, EventPoints)

\subsubsection{UNF to 3NF}
\underline{UNF}


\begin{tabular}{l l l}
EventID        & Age              & Date               \\
EventPoints    & EventType        & EventPointType     \\
ClubReferance  & RiderID          & Forename           \\
Surname        & HandicapMod      & EventPointsID      \\
Club           & RecordID         & RideTime           \\
CourceID       & CourceCode       & CourceDistance     \\
DateJoined     & DateLeft         & EventReferanceID   \\
EventTypeID    & ClubID           & Laps               \\

\end{tabular}

\underline{1NF}

\begin{tabular}{|l l|l|}
\hline
REPEATING           &                 & NON-REPEATING       \\ \hline
\underline{RecordID}& RideTime        & \underline{EventID} \\ \hline
\underline{EventID} & EventPointsType & Date                \\ \hline
ClubReferance       & DateLeft        & CourseID            \\ \hline
Surname             & HandicapMod     & CourseCode          \\ \hline
DateJoined          & RiderID         & CorseDistance       \\ \hline 
EventPointsID       & Age             & EventReferanceID    \\ \hline
EventTypeID         & ClubID          & Laps                \\ \hline
\end{tabular}




\underline{2NF}

\begin{tabular}{|l|l|}
\hline
\underline{RecordID} & \underline{EventID} \\ \hline
Forename             & Date                \\ \hline
Surname              & CourceID            \\ \hline
ClubReferance        & CourceCode          \\ \hline
Club                 & CourceDistance      \\ \hline
DateJoined           & EventTypeID         \\ \hline
DateLeft             & EventType           \\ \hline
ClubID               & EventReferanceID    \\ \hline
                     & Laps                \\ \hline
\underline{RecordID} &                     \\ \hline
\underline{EventID}  &                     \\ \hline 
EventPointsID        &                     \\ \hline
RideTIme             &                     \\ \hline
Age                  &                     \\ \hline
HandicapMod          &                     \\ \hline
EventPoints          &                     \\ \hline
EventPointsType      &                     \\ \hline
\end{tabular}

\underline{3NF}

\begin{tabular}{|l|l|l|l|}
\hline
\begin{tabular}{l}
	\\
	\underline{EventID}     \\
	\emph{CourseID}         \\
	\emph{EventReferanceID} \\
	Date                    \\
	Laps                    \\
	\\
\end{tabular}& \begin{tabular}{l}
	\underline{CourseID} \\
	CourseCode           \\
	CourseDistance       \\
\end{tabular} & \begin{tabular}{l}
	\underline{EventReferanceID} \\
	\emph{EventTypeID}           \\
\end{tabular} &\begin{tabular}{l}
	\underline{RiderID} \\
	Forename            \\
	Surname             \\
\end{tabular} \\ \hline
\begin{tabular}{l}
	\underline{ClubReference} \\
	\emph{RiderID}            \\
	\emph{ClubID}             \\
	DateJoined                \\
	DateLeft                  \\
\end{tabular} & \begin{tabular}{l}
	\\
	\underline{RecordID} \\
	\emph{EventID}       \\
	\emph{RiderID }      \\
	RideTime             \\
	Age                  \\
	HandicapMod          \\
	\\
\end{tabular} & \begin{tabular}{l}
	\underline{EventPointsID} \\
	\emph{RecordID}           \\
	EventPointsType           \\
	EventPoints               \\
\end{tabular} & \begin{tabular}{l}
	\underline{ClubID} \\
	Club               \\
\end{tabular} \\ \hline
\begin{tabular}{l}
	\\
	\underline{EventTypeID} \\
	EventType               \\
	\\
\end{tabular} & & & \\ \hline 
\end{tabular}

\section{Security and Integrity of the System and Data}

\subsection{Security and Integrity of Data}

\subsection{System Security}

\section{Validation}

\section{Testing}

\begin{landscape}
\subsection{Outline Plan}

\begin{center}
    \begin{tabular}{|p{2cm}|p{5cm}|p{5cm}|p{4cm}|}
        \hline
        \textbf{Test Series} & \textbf{Purpose of Test Series} & \textbf{Testing Strategy} & \textbf{Strategy Rationale}\\ \hline
        Example & Example & Example & Example \\ \hline
    \end{tabular}
\end{center}

\subsection{Detailed Plan}

\begin{center}
    \begin{longtable}{|p{1.5cm}|p{2.5cm}|p{2.5cm}|p{2cm}|p{2cm}|p{2cm}|p{2cm}|p{2cm}|}
        \hline
        \textbf{Test Series} & \textbf{Purpose of Test} & \textbf{Test Description} & \textbf{Test Data} & \textbf{Test Data Type (Normal/ Erroneous/ Boundary)} & \textbf{Expected Result} & \textbf{Actual Result} & \textbf{Evidence}\\ \hline
        Example & Example & Example & Example & Example & Example & Example & Example \\ \hline
    \end{longtable}
\end{center}
\end{landscape}
