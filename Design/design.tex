\chapter{Design}

\section{Overall System Design}

\subsection{Short description of the main parts of the system}

\subsection{System flowcharts showing an overview of the complete system}

\section{User Interface Designs}

\section{Program Structure}

\subsection{Top-down design structure charts}

\subsection{Algorithms in pseudo-code for each data transformation process}

\subsection{Object Diagrams}

\subsection{Class Definitions}

\section{Prototyping}

\section{Definition of Data Requirements}

\subsection{Identification of all data input items}

\subsection{Identification of all data output items}

\subsection{Explanation of how data output items are generated}
Blank
\subsection{Data Dictionary}

\begin{tabular}{|p{1.5cm}|p{1.5cm}|l|l|l|p{2.5cm}|}
	\hline
	NAME & DATA TYPE & LENGTH & VALIDATION & EXAMPLE DATA & APPROXIMATE SIZE \\ \hline
	EventID & Integer & 0-9999 & Range & 10 & 4 Bytes \\ \hline
	CourseID & Integer & 0-9999 & Range & 10 & 4 Bytes \\ \hline
	Date & Date & DD/MM/YYYY & Range & 1 & 3 Bytes \\ \hline
	CircuitSeries & Boolean & 0 or 1 & Range & 1 & 2 Bytes \\ \hline
	Handicap10 & Boolean & 0 or 1 & Range & 1 & 2 Bytes \\ \hline
	Handicap25 & Boolean & 0 or 1 & Range & 1 & 2 Bytes \\ \hline
	HillClimb & Boolean & 0 or 1 & Range & 1 & 2 Bytes \\ \hline
	Transmedia & Boolean & 0 or 1 & Range & 1 & 2 Bytes \\ \hline
	Juvenile & Boolean & 0 or 1 & Range & 1 & 2 Bytes \\ \hline
	Code & String & 2 Characters & length & TR & 2 Bytes \\ \hline
	RiderID & Integer & 0-9999 & Range & 10 & 4 Bytes \\ \hline
	Forename & String & 50 Characters & length & Peter & 50 Bytes \\ \hline
	Surname & String & 50 Characters & length & Millard & 50 Bytes \\ \hline
	TCID & Integer & 0-9999 & Range & 10 & 4 Bytes \\ \hline
	Handicap10Points & Integer & 0-350 & Range & 125 & 4 Bytes \\ \hline
	CircuitPoints & Integer & 0-350 & Range & 125 & 4 Bytes \\ \hline
	TransmediaPoints & Integer & 0-350 & Range & 125 & 4 Bytes \\ \hline
	JuvenlePoints & Integer & 0-350 & Range & 125 & 4 Bytes \\ \hline
	CourseID & Integer & 0-9999 & Range & 10 & 4 Bytes \\ \hline
	RecordID & Integer & 0-9999 & Range & 10 & 4 Bytes \\ \hline
	RideTime & Time & HH:MM:SS & Range & 01:25:23 & 3 Bytes \\ \hline
	HandicapTime & Time & HH:MM:SS & Range & 00:25:23 & 3 Bytes \\ \hline
	RacePosition & Integer & 1-50 & Range & 05 & 4 Bytes \\ \hline
	Club & String & 50 Characters & Length & Team Cambridge & 100 Bytes \\ \hline
	Age & Integer & 12 – 99 & Range & 18 & 4 Bytes \\ \hline
\end{tabular}

\subsection{Identification of appropriate storage media}

\section{Database Design}

\subsection{Normalisation}

\subsubsection{ER Diagrams}
\begin{figure}[H]
    \includegraphics[width=\textwidth]{./ER/ERDesing.jpg}
\end{figure}

\subsubsection{Entity Descriptions}
After looking the specification, I found that there was attributes missing from the entity that needed to be included so I modified the entity descriptions from the analysis.

Event(\underline{EventID}, \emph{CourseID}, Date, CircuitSeries, Handicap10, Handicap25, HillClimb, Transmedia, Juvenile, Code)

Rider(\underline{RiderID}, Forename, Surname)

TCRecord(\underline{TCID},\emph{RecordID}, Handicap10Points, CircuitPoints, TransmediaPoints, JuvenlePoints)

Record(\underline{RecordID}, \emph{RiderID},\emph{EventID}, RideTime, HandicapTime, RacePosition, Club, Age)

\subsubsection{UNF to 3NF}

\underline{UNF}
\begin{tabular}{l l l}
EventID & CourseID & Date //
CircuitSeries & Handicap10 & Handicap25 //
HillClimb & Transmedia & Juvenile //
Code & RiderID & Forename //
Surname & TCID & Handicap10Points //
CircuitPoints & TransmediaPoints & JuvenlePoints //
CourseID & RecordID & RideTime //
HandicapTime & RacePosition Club //
Age & & //
\end{tabular}}

\underline{1NF}

\begin{tabular}{|l|l|}
\hline
REPEATING           & NON-REPEATING \\ \hline
\underline{RiderID} & \underline{EventID} \\ \hline
\emph{EventID}      & Date \\ \hline
Forename             & CircuitSeries \\ \hline 
Surname             & Handicap10 \\ \hline 
TCID                & Handicap25 \\ \hline
Handicap10Points    & HillClimb \\ \hline 
CircuitPoints       & Transmedia \\ \hline
TransmediaPoints    & Juvenile \\ \hline
JuvenilePoints      & Code \\ \hline
RecordID            & \\ \hline
RideTime            & \\ \hline
HandicapTime        & \\ \hline
RacePosition        & \\ \hline
Club                & \\ \hline
Age                 & \\ \hline

\end{tabular}

\underline{2NF}

\begin{tabular}{|l|l|}
\hline
REPEATING & NON-REPEATING \\ \hline
\underline{RiderID} & \underline{EventID} \\ \hline
Forename & Date \\ \hline
Surname & CircuitSeries \\ \hline 
\underline{TCID} & Handicap10 \\ \hline 
Handicap10Points & Handicap25 \\ \hline
CircuitPoints & HillClimb \\ \hline 
TransmediaPoints & Transmedia \\ \hline
JuvenilePoints & Juvenile \\ \hline
\underline{RecordID} & Code \\ \hline
RideTime & \\ \hline
HandicapTime & \\ \hline
RacePosition & \\ \hline
Club & \\ \hline
Age & \\ \hline

\end{tabular}

\underline{3NF}

\begin{tabular}{|l|l|l|l|}
\hline
\begin{tabular}{l}
\underline{RiderID}\\
Forename \\
Surname \\
\end{tabular} & \begin{tabular}{l}
\underline{EventID} \\
Date \\
CircuitSeries \\
Handicap10 \\
Handicap25 \\
HillClimb \\
Transmedia \\
Juvenile \\
Code \\
\end{tabular} & \begin{tabular}{l}
\underline{TCID} \\
Handicap10Points \\
CircuitPoints \\
TransmediaPoints \\
JuvenilePoints \\
\end{tabular} & \begin{tabular}{l}
\underline{RecordID} \\
RideTime \\
HandicapTime \\
RacePosition \\
Club \\
Age \\
\end{tabular}\\
\hline
\end{tabular}

\section{Security and Integrity of the System and Data}

\subsection{Security and Integrity of Data}

\subsection{System Security}

\section{Validation}

\section{Testing}

\begin{landscape}
\subsection{Outline Plan}

\begin{center}
    \begin{tabular}{|p{2cm}|p{5cm}|p{5cm}|p{4cm}|}
        \hline
        \textbf{Test Series} & \textbf{Purpose of Test Series} & \textbf{Testing Strategy} & \textbf{Strategy Rationale}\\ \hline
        Example & Example & Example & Example \\ \hline
    \end{tabular}
\end{center}

\subsection{Detailed Plan}

\begin{center}
    \begin{longtable}{|p{1.5cm}|p{2.5cm}|p{2.5cm}|p{2cm}|p{2cm}|p{2cm}|p{2cm}|p{2cm}|}
        \hline
        \textbf{Test Series} & \textbf{Purpose of Test} & \textbf{Test Description} & \textbf{Test Data} & \textbf{Test Data Type (Normal/ Erroneous/ Boundary)} & \textbf{Expected Result} & \textbf{Actual Result} & \textbf{Evidence}\\ \hline
        Example & Example & Example & Example & Example & Example & Example & Example \\ \hline
    \end{longtable}
\end{center}
\end{landscape}
