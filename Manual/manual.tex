\chapter{User Manual}

\section{Introduction}
This system is disned for cycling clubs that host time triles and want a digital system for storig results and calculation handicaps. Spsifcly this system is disined for Team Cambridge as the handicap calculation is sepsific to there handicap system. The handicap module can be replaced so that any other club can use the system with there own handicap system.
\section{Installation}

\subsection{Prerequisite Installation}

%include as many subsubsections as necessary for each piece of required software
\subsubsection{Installing Python}
\begin{enumerate}
\item In an internet browser go to the python web site found at https://www.python.org/
\item Click on the downloads tab
\item Under "Download the latest version for windows" click on the Download Python 3.x.x button
\item Run the .msi file that you have just downloaded, and follow the on screen instructions
\item Python 3 should now be installed
\end{enumerate}
\subsubsection{Installing PyQt}
\begin{enumerate}
\item In an internet browser go to the river bank computing web site found at http://www.riverbankcomputing.co.uk/
\item Move your cursor over the software tab, then down to the PyQt tab, then click on the on the "PyQt 4 Download" button
\item Scroll down to the "Binary Packages" section and click on the "PyQt4-4.11.3-gpl-Py3.4-Qt4.8.6-x32.exe	" link
\item Run the .exe file that just downloaded and follow the on screen instructions
\item PyQt should now be installed
\end{enumerate}
\subsection{System Installation}
\begin{enumerate}
\item In an internet browser go to\newline https://www.dropbox.com/s/ckklkznbh5nfq6t/TeamCambridgeDatabaseManager.zip?dl=0
\item Then click on the download button
\item open the downloaded file in windows explorer, then click on the "extract all files" button in the top left hand of the window
\item then select a directory where you wish the application to be installed, then click extract
\item Browse to the location that you have just chosen, and run the  DatabaseConstructor.py file
\end{enumerate}
\subsection{Running the System}
To run the system you will need to run the MainWindow.py file
\section{Tutorial}

\subsection{Introduction}

\subsection{Assumptions}
I am assuming that the user has a basic understadning of the time tile system and basic skills with a computor.
\subsection{Tutorial Questions}

\subsubsection{What is the process of loading the database}
When you run the MainWindow.py file a file dialog winodw will open, navigate to the file where the TeamCambridge.db file is loacted and dobble clik on it. the database should now be laoded.

If yo missclick on another file then you will need to close the aplication and start again.

\subsubsection{What is the process of adding a rider?}
\begin{enumerate}
\item Navgiate to the Rider table by clicking on the "Tables" button on the menu bar, then click the "Rider" button
\item Now click on the "Add Data" button
\item Fill the "Forename" and "Surname" line edits with apropet data and click the "OK" button
\item The data will be added to the database and the database will be saved automaticaly.
\end{enumerate}

Note: If one or both of the fields are not filled an error messages will apper when the "OK" button is pressed

\subsubsection{What is the process of adding a club?}
\begin{enumerate}
\item Navgiate to the Club table by clicking on the "Tables" button on the menu bar, then click the "Club" button
\item Now click on the "Add Data" button
\item Fill the "Club" line edit with apropet data and click the "OK" button
\item The data will be added to the database and the database will be saved automaticaly.
\end{enumerate}

Note: If  the field is not filled an error messages will apper when the "OK" button is pressed

\subsubsection{What is the process of adding a course?}
\begin{enumerate}
\item Navgiate to the Course table by clicking on the "Tables" button on the menu bar, then click the "Course" button
\item Now click on the "Add Data" button
\item Fill the "Course Code" and "Course Distance" line edits with apropet data and click the "OK" button
\item The data will be added to the database and the database will be saved automaticaly.
\end{enumerate}

Note: If  any of the fields are not filled an error messages will apper when the "OK" button is pressed

\subsubsection{What is the process of connecting a club to a rider?}

\subsubsection{What is the process of adding an event?}

\subsubsection{What is the process of adding a record to an event?}

\subsubsection{What is the process of adding an event with records and points from begining to end?}
\begin{enumerate}
\item First you will need to add a Course(see above for detailes)(if the course you want to use is allready added to the databse then skip this instruciton)
\item Second you will need to add an Event and link it to the course you have just made(see above for detaules)
\item 
\end{enumerate}

\subsection{Saving}
The system saves to the TeamCambridge.db file each time data is added or deleted from the database, so you will not need to save the database befor closing the aplicaiton.
\subsection{Limitations}

\section{Error Recovery}

%include as many subsections as necessary for each error
\subsection{Error 1}

\subsection{Error 2}

\section{System Recovery}

\subsection{Backing-up Data}

\subsection{Restoring Data}
