\chapter{User Manual}

\section{Introduction}

\section{Installation}

\subsection{Prerequisite Installation}

%include as many subsubsections as necessary for each piece of required software
\subsubsection{Installing Python}
\begin{enumerate}
\item In an internet browser go to the python web site found at https://www.python.org/
\item Click on the downloads tab
\item Under "Download the latest version for windows" click on the Download Python 3.x.x button
\item Run the .msi file that you have just downloaded, and follow the on screen instructions
\item Python 3 should now be installed
\end{enumerate}
\subsubsection{Installing PyQt}
\begin{enumerate}
\item In an internet browser go to the river bank computing web site found at http://www.riverbankcomputing.co.uk/
\item Move your cursor over the software tab, then down to the PyQt tab, then click on the on the "PyQt 4 Download" button
\item Scroll down to the "Binary Packages" section and click on the "PyQt4-4.11.3-gpl-Py3.4-Qt4.8.6-x32.exe	" link
\item Run the .exe file that just downloaded and follow the on screen instructions
\item PyQt should now be installed
\end{enumerate}
\subsection{System Installation}
\begin{enumerate}
\item In an internet browser go to\newline https://www.dropbox.com/s/ckklkznbh5nfq6t/TeamCambridgeDatabaseManager.zip?dl=0
\item Then click on the download button
\item open the downloaded file in windows explorer, then click on the "extract all files" button in the top left hand of the window
\item then select a directory where you wish the application to be installed, then click extract
\item Browse to the location that you have just chosen, and run the  DatabaseConstructor.py file
\item To run the system you will need to run the MainWindow.py file
\end{enumerate}
\subsection{Running the System}

\section{Tutorial}

\subsection{Introduction}

\subsection{Assumptions}

\subsection{Tutorial Questions}

%include as many subsubsections as necessary for each question in your list
\subsubsection{Question 1}

\subsubsection{Question 2}

\subsection{Saving}

\subsection{Limitations}

\section{Error Recovery}

%include as many subsections as necessary for each error
\subsection{Error 1}

\subsection{Error 2}

\section{System Recovery}

\subsection{Backing-up Data}

\subsection{Restoring Data}